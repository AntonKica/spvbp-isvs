\chapter{Ľahký úvod do situácie}
%Neoddeliteľnou súčasťou nášho každodenného života je informácia. Informáciu v súčasnej dobe uchovávame, spracovávame a 
%šírime najmä v digitálnej forme. Práve digitálne informačné a komunikačné technológie (ďalej d-IKT) nám pomáhajú 
%zaobchádzať s enormným množstvom informácií vo svete. Pre jemnú ilustráciu, jedna štatistika \cite{STATISTA} odhadla, 
%že v roku 2020 sa vo svete nachádzalo vyše 79 ZB dát (ZB -- zetabajt, $1$ ZB = $10^9$ TB).


%Aj samotná Európska únia (ďalej EÚ) sa snaží šíriť d-IKT a posilniť digitalizáciu členských štátov. EÚ vypracovala
%digitálny rámec pod názvom \say{Europe’s Digital Decade}, vrámci ktorého boli vytýčené spoločné ciele digitálnej 
%transformácie do roku 2030.

%Od roku 2014 monitoruje Európska komisia digitálny rozvoj členských štátov pomocou Indexu digitálnej ekonomiky a
%spoločnosti
%\cite{DESI22}. V skratke, tento index profiluje štáty v štyroch oblastiach: pripojiteľnosť; ľudský kapitál; 
%využívanie internetových služieb; integrácia digitálnych technológií.

%Určite by pre nikoho nebolo prekvapením, že najväčší rozvoj v digitalizácií prebehol v roku 2020 - keď nastúpila pandémia COVID-19.
%Priamo z výročnej správy DESI\cite{DESI22} o roku 2020:
%\begin{quote}
%  Businesses provided more fully digitised products and
%  services: 34\% before the Covid-19 crisis and 50\% during the pandemic6; and bought more cloud
%  computing services: 24\% before the pandemic in 2019 and 41\% in 2021. Significant differences
%  continue to persist between large enterprises and SMEs\textsuperscript{8}, given that 72\% of large enterprises
%  subscribed to cloud computing services compared to 40\% of SMEs.
%\end{quote}

%Pokrok v tomto smere sa v nasledujúcich rokoch znížil. Z indexu DESI\cite{DESI22} za rok 2022  má Slovensko skóre 43,4
%(oproti európskemu priemeru 52,3), v poradí 24. (prvé Fínsko 69.6; posledné Rumunsko 30.6) spomedzi 27 štátov. Zaostávame.

%\section{Kybernetická a informačná bezpečnosť}
%Najprv si uvedieme základne pojmy podľa Malého výkladového slovníka \cite{KUDIaKB-MVS}: \\

%\textbf{kybernetický priestor}
%\begin{quote}
%  globálny dynamický otvorený systém, ktorý tvoria
%  telekomunikačné a počítačové siete,
%  informačné a komunikačné systémy, ich
%  programové vybavenie a údaje, ktoré sa
%  pomocou nich spracovávajú; ...
%\end{quote}


%\textbf{kybernetická a informačná bezpečnosť} (ďalej KIB)
%\begin{quote}
%  systém opatrení na zaistenie odolnosti
%  kybernetického priestoru, ako aj činností a
%  prostriedkov zameraných na dosiahnutie
%  požadovanej úrovne bezpečnosti prvkov
%  kybernetického priestoru vrátane riešenia
%  incidentov a následných opatrení a činností
%\end{quote}

%\textbf{aktívum}
%\begin{quote}
%  čokoľvek, čo má pre organizáciu hodnotu, čokoľvek, čo je: 
%    hmotné (zariadenia, personál, ...) alebo nehmotné (peniaze, informácia, ...),
%    môže sa stať objektom útoku a
%    vyžaduje si ochranu
%\end{quote}

%\section{d-IKT na Slovensku}
%Rôzne štátne inštitúcie, firmy, banky, poisťovne... využívajú d-IKT na svoju prevádzku. Pomocou d-IKT spravujú svoje 
%systémy. V týchto systémoch sa nachádzajú mnohé aktíva. Tieto aktíva sú vystavené kybernetickému priestoru. 
%V kybernetickom priestore sú prítomné rôzne hrozby -- prírodné, technické, ľudské.
%Potrebujeme d-IKT, ich systémy a ich aktíva chrániť pred poruchami, výpadkami, chybami, odcudzením alebo útokmi v 
%kybernetickom priestore.

%\section{Potreba KIB}
%Chrániť informácie nachádzajúce sa v systémoch verejnej správy nám, mimo morálnej potreby, ukladá jednak legislatíva 
%Slovenska, no tiež legislatíva EÚ - mnoho z nás sa určite stretlo s nariadením GDPR (General Data Protection Regulation). 
%Na Slovensku najprominentnejšie zákony a vyhlášky týkajúce sa KIB sú najmä:

%\begin{itemize}
%  \item Zákon o kybernetickej a informačnej bezpečnosti 69/2018
%  \item Vyhláška NBÚ č. 362/2018
%  \item Zákon o informačných technológiách verejnej správy (ITVS) 95/2019
%  \item Vyhláška podpredsedu vlády 179/2020
%\end{itemize}

%Táto legislatíva kladie požiadavky KIB na informačné služby verejnej správy (ďalej ISVS). Požiadavky na 
%vypracovanie bezpečnostného projektu, dokumentácie, riadenie KIB, analýzy a riadenia rizík, atď. pre ISVS. 

%  \section{Súčasný stav KIB na Slovensku}
%Národný bezpečnostný úrad (ďalej NBÚ) publikuje každoročne výročnú správu o KIB v Slovenskej republike. Spomedzi mnohé 
%sa výročná správa venuje prehľadom globálnych hrozieb, incidentov, trendom a významným incidentom. Vykonáva bezpečnostné
%audity v oblastiach: bankovníctvo, zdravotníctvo, energetika, infraštruktúra a verejná správa.

%Podľa najnovšej výročnej správy \cite{KIBSK22} došlo za rok 2022 k $8'887'103$ potenciálnym bezpečnostným incidentom - 
%naprieč všetkými sektormi. Keď sa zúžime na sektore verejnej správy (čo je predmetom našej bakalásrkej práce), audit v zistil
%najmä tieto nedostatky: 

%\begin{quote}
%  \begin{itemize}
%    \item nebol preukázaný systém riadenia kybernetickej bezpečnosti,
%    \item bezpečnostná stratégia kybernetickej bezpečnosti ani ďalšia bezpečnostná dokumentácia nebola predložená,
%    \item manažér kybernetickej bezpečnosti nie je formálne menovaný, je v konflikte záujmov, má nevhodne kumulované zodpovednosti,
%    \item analýza rizík nie je zakotvená ako proces v interných predpisoch ani metodicky popísaná, nevykonáva sa,
%    \item v organizácii sa nachádzajú vysoko privilegované účty, ktoré sú spoločné a nemajú definovaných vlastníkov a účel,
%    \item v organizácii neexistuje definícia závažného kybernetického bezpečnostného incidentu,
%    \item nevypracovala postupy a nemá dostatočné schopnosti na detekciu, zvládanie a poučenie sa z prípadných incidentov.
%  \end{itemize}
%\end{quote}


