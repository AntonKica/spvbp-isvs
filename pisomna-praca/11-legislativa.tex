\chapter{Kybernetická a informačná bezpečnosť v legislatíve}
Kybernetická a informačná bezpečnosť je obsiahnutá v niekoľkých zákonoch, resp. vyhláškach. V niektorých je KIB 
spomenutá okrajovo. My sa sústredíme na tie zákony a vyhlášky, ktoré kladú veľký dôraz na KIB v organizácií. 
Rozoberieme postupne 69/2018 Z.z.\cite{69/2018}, 362/2018 Z.z\cite{362/2018}, 95/2019 Z.z\cite{95/2019} a 
179/2020 Z.z\cite{179/2020}. Pokúsime sa z nich vytýčiť tie najrelevantnejšie informácie, keďže sú vcelku rozsiahle.

% src: https://tex.stackexchange.com/a/10012
\newcommand\lawsubtitle[1]{\begin{flushleft}\small\textit{#1}\end{flushleft}}

\section{Zákon č. 69/2018 Z. z.}
\lawsubtitle{Zákon o kybernetickej bezpečnosti a o zmene a doplnení niektorých zákonov \cite{69/2018}}

V rámci tohto zákona sú vymedzené rôzne právomoci, kompetencie a povinnosti Národného bezpečnostného úradu (ďalej NBÚ).
NBÚ spravuje štandardy, určuje národnú stratégiu KIB, spravuje jednotky pre riešenie KIB incidentov \footnote{ Tiež
známe ako CSIRT, z angl. \textit{computer security incident response team} }, vykonáva audity, odníma certifikáty KIB, atď.

Tento zákon tiež vymedzuje minimálne požiadavky na zabezpečenie KIB, pričom nezasahuje do žiadnych detailov. 
Tieto požiadavky sú kladené na služby (a ich prevádzkovateľov), ktoré sú rozdelene na dve kategórie: 
základná služba a digitálna služba.

\textbf{Základnou službou} sa rozumie napr. prvok kritickej infraštruktúry alebo sektor bankovníctva, dopravy, 
energetika, verejná správa a mnohé ďalšie (NBÚ udržiava kompletný zoznam), vrátane ich podsektorov. Tiež, služba 
závislá od základnej služby je považovaná za základnú službu. Poskytovateľom základnej služby je orgán verejnej 
moci alebo fyzická osoba.

\textbf{Digitálnou službou} sa rozumie online trhovisko, internetový vyhľadávač alebo služba v oblasti cloud computingu. 
Poskytovateľom digitálnej služby je veľká organizácia aspoň o 50 zamestnancoch a ročným obratom vyše 10 mil. eur.

Nakoniec definuje 16 oblasti, pre ktoré sa majú realizovať bezpečnostné opatrenia: 
\begin{enumerate}[a)]
    \item organizácie kybernetickej bezpečnosti a informačnej bezpečnosti,
    \item riadenia rizík kybernetickej bezpečnosti a informačnej bezpečnosti,
    \item personálnej bezpečnosti,
    \item riadenia prístupov,
    \item riadenia kybernetickej bezpečnosti a informačnej bezpečnosti vo vzťahoch s tretími stranami,
    \item bezpečnosti pri prevádzke informačných systémov a sietí,
    \item hodnotenia zraniteľností a bezpečnostných aktualizácií,
    \item ochrany proti škodlivému kódu,
    \item sieťovej a komunikačnej bezpečnosti,
    \item akvizície, vývoja a údržby informačných sietí a informačných systémov,
    \item zaznamenávania udalostí a monitorovania,
    \item fyzickej bezpečnosti a bezpečnosti prostredia,
    \item riešenia kybernetických bezpečnostných incidentov,
    \item kryptografických opatrení,
    \item kontinuity prevádzky,
    \item auditu, riadenia súladu a kontrolných činností.
\end{enumerate}
Avšak, tento zákon nezachádza do žiadnych detailov bezpečnostných opatrení.   

\section{Vyhláška č. 362/2018 Z. z.}
\lawsubtitle{Vyhláška Národného bezpečnostného úradu, ktorou sa ustanovuje obsah bezpečnostných opatrení, obsah a 
štruktúra bezpečnostnej dokumentácie a rozsah všeobecných bezpečnostných opatrení \cite{362/2018}}

Táto vyhláška vykonáva predpis z predchádzajúceho zákona (69/2020 Z.z. \cite{69/2018}). Detailnejšie špecifikuje obsah 
bezpečnostných opatrení a štruktúru bezpečnostnej dokumentácie. Definuje stratégiu KIB, klasifikáciu a kategorizáciu 
informačných systémov do troch kategórií. Detailnejšie popisuje požiadavky na bezpečnostné opatrenia všetkých 16 
oblasti zo zákona 69/2020 Z.z. \cite{69/2018}. Napriek špecifickým požiadavkám neposkytuje ani nereferencuje žiadne 
návody pre dosiahnutie stanovených cieľov.

\section{Zákon č. 95/2019 Z. z.}
\lawsubtitle{ Zákon o informačných technológiách vo verejnej správe a o zmene a doplnení niektorých zákonov \cite{95/2019}}
Tento zákon ustanovuje organizáciu správy a základné požiadavky kladené na ITVS. Definuje pojmy:
% TODO hodí sa to sem alebo ani nie
\begin{itemize}
  \item \textbf{informačná technológia}, prostriedok alebo postup slúžiaci na spracovávanie údajov alebo informácií 
    v elektronickej podobe, \item \textbf{informačná služba}, funkčný celok zabezpečujúci informačnú činnosť 
      prostredníctvom technických a technologických prostriedkov.
\end{itemize}
Správu ITVS vykonáva orgán vedenia (MIRRI) a orgán riadenia (ministerstvo, Generálna prokuratúra, obec a vyšší územný 
celok, právnická osoba, atď.). Zodpovedný za správu ITVS je \textbf{správca}. Správca je (mimo iné) povinný:
\begin{enumerate}[-]
    \item riadiť sa všeobecne uznávanými štandardami riadenia IT a metodickými usmerneniami orgánu vedenia,
    \item zabezpečiť riadenie rizík, riadenie bezpečnosti, identifikovať a udržiavať zoznám aktív a
    \item zaviesť systém riadenia informačnej bezpečnosti (ISMS) a vypracovať bezpečnostný projekt pre všetky ISVS.
\end{enumerate}
% TODO: skús to neuseknúť

\section{Vyhláška č. 179/2020 Z. z.}
\lawsubtitle{Vyhláška Úradu podpredsedu vlády Slovenskej republiky pre investície a informatizáciu, ktorou sa ustanovuje 
spôsob kategorizácie a obsah bezpečnostných opatrení informačných technológií verejnej správy \cite{179/2020}}

Táto vyhláška vykonáva predpis z predchádzajúceho zákona (95/2019 Z.z. \cite{95/2019}). Ustanovuje kategorizáciu a 
podrobne určuje obsah bezpečnostných opatrení ITVS. Rozdeľuje minimálne bezpečnostné opatrenia požadované od ITVS 
do troch kategórií:
\begin{enumerate}[-]
  \item \textbf{kategória I}, napr. obec do 6000 obyvateľov, právnická osoba,
  \item \textbf{kategória II}, napr. obec nad 6000 obyvateľov (mimo krajské mestá), 
    prevádzkovateľ základných služieb kategórie I a II (podľa 362/2018 Z.z. \cite{362/2018}) a
  \item \textbf{kategória III}, napr. krajské mesto, samosprávny kraj, rôzne úrady a kancelárie, 
    Sociálna a zdravotná poisťovna, prevádzkoveteľ základných služieb kategórie III (podľa 362/2018 Z.z. \cite{362/2018}).
\end{enumerate}
Pre každú z kategórií sú v prílohe detailne enumerované minimálne bezpečnostné požiadavky (I - má najmenej, III. najviac).

V tejto vyhláške je tiež explicitné vyhlásene, že vychádza z niekoľkých štandardov\footnote{ Konkrétne sú to štandardy 
ISO/IEC 2700*}. Dáva však voľnosť použiť aj iné (ekvivalentné) štandardy, pomocou ktorých sa dosiahne rovnaký výsledok 
bezpečnostných opatrení.

\section{Zhrnutie legislatívy}
Z hľadiska KIB zákony a vyhlášky vyžadujú od služieb, resp. ITVS aby realizovali všeobecné bezpečnostné opatrenia v
niekoľkých oblastiach. Služby rozdeľujú do rôznych úrovní kategórií, ktoré sa líšia rozsahom minimálnych 
bezpečnostných opatrení. Správcom týchto služieb ukladajú povinnosť vypracovať systém riadenia bezpečnosti (ISMS), 
pričom pri plnení svojich zákonných povinností majú postupovať pomocou všeobecne uznávaných štandardov a dobrej praxe.
