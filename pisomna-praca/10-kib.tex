\chapter{Informačná bezpečnosť a Európska únia}
Potenciál digitálnych informačných a komunikačných technológií (d-IKT)\footnote{Zariadenia, ktoré zaznamenávajú, spracovávajú
a prenášajú informácie v digitálnej forme.} je Európskej únii (EÚ) dlho známy. V roku 2010, tri roky po vypuknutí Veľkej
recesie, predostrela EÚ digitálnu agendu pre Eúropu\cite{DAEU}. V tejto agende bol zdôraznený význam d-IKT pre cieľe EÚ.
Cieľom EÚ bolo vymániť sa z krízy a pripraviť sa na nasledujúce desaťročie. 

Neskôr, v roku 2015, bola táto agenda rozšírená Stratégiou pre jednotný digitálny trh\cite{DMA}. Zjednotením digitálneho 
trhu chcela EÚ dosiahnúť voĺný prístup k digitálnemu tovaru, podporiť podmienky digitálnych sieti a služieb a zvýsiť rast 
potenciálu digitálneho hospodárstva. Dôraz bol kladený aj na vysokú ochranu osobných údajov a spotrebiteľov, bez ohľadu
na štátnu príslušnosť alebo miesto pobytu.

Prvá digitálna agenda sa dočkala aj úspechov. Asi jeden z najznámejších výsledkov je nariadenie (EÚ) 2016/679 o ochrane
súkromia a osobných údajov - GDPR (z angl. \textit{General Data Protection Regulation}). V roku 2017 boli zrušené dodatočné 
poplatky za roaming pri cestovaní do inej krajiny EÚ - "Roam Like At Home"\cite{RoamingEU}.
Ďalej v roku 2018 vyšlo nariadenie (EÚ) 2018/302, ktoré riešilo bezdôvodné geografické blokovanie a diskriminovanie
\footnote{Napríklad z dôvodu štátnej príslušnosti, miesta bydliska, a pod.}; nariadenie (EÚ) 2018/1807 o voľnom toku neosobných
informácií\footnote{Rôzne spoločnosti a orgány verejnej správy môžu tieto informácie uchovávať a spracovávať prakticky
kdekoľvek.}.

Digitálny pokrok členských štátov monitoruje Európska komisia pomocou indexu digitálnej ekonomiky a spoločnosti (Digital
Economy and Society Index, DESI). Sledovaný je pokrok v smeroch: zručností, digitálna transformáca podnikov, bezpečná a
udržateľná digitálna infraštruktúra a digitalizáca verejných služueb \footnote{Štyri cieľe digitálného kompasu Európskej
digitálnej dekády.}. V správe za rok 2023 \cite{Report2023}
má Slovensko mnohé nedostatky: takmer polovica obyvateľstva nemá digitálne zručnosti; občania a podniky čelia ťažkostiam 
pri používaní verejných služieb\footnote{Najmä nízka použiteľnosť a transparentnosť.}; dostupnosť elektronických zdravotných
záznamov.



\section{Potreba chrániť d-IKT}
EǓ si tiež uvedomala potrebu chŕaniť d-IKT, ktoré sa stali kritickou infraštruktúrou\footnote{Infraštruktúra, ktorej naruśenie
alebo nefunkčenie spôsobí pretrávajúci nedostatok dodávok, narušenie verejnej bezpečnosti alebo iné signfikantné následky.}
spoločnosti. Vznikla prirodzená potreba chrániť \footnote{Odhadované škody kybernetickej kriminality sú €5.5 triliónov k
roku 2021 [citation needed]} základné služby ako nemocnice, energetické siete, železnice a pod. V Stratégii kybernetickej
bezpečnosti \cite{CSS} opísala EÚ, ako chrániť a posilniť prvky kritickej infraštruktúry. Stratégia sa sústredila na odolnosť,
kapacitu predchádzať, odradiť, reagovať a vzájomnú spoluprácu pred bezpečnostnými rizikami a hrozbami
d-IKT.

V roku 2016 bola zavedená direktíva sieťovej a informačnej bezpečnosti (Network and Information Security, NIS), ktorá
bola pre nedostatky v roku 2023 nahrádená rozšíreniou direktivou NIS2\footnote{Pre prehľadnosť budeme využívať skratku
NIS2 namiesto NIS.}. NIS2 stanovila opatrenia pre spoločný vysoký level kybernetickej bezpečnosti. Členské štáty by mali
byť adekvátne vybavené technologicky a kapacitne, aby vedeli vhodne predchádzať, detegovať, reagovať a zmiernovať risk.
Direktíva tiež ukladá povinnosť adoptovať politiku aktívnej kybernetickej bezpečnosti, zaviesť národnu kybernetickú stratégiu
pre malé a stredné podniky a zriadiť si jednotky CSIRT\footnote{Jednotky CSIRT majú za úlohu riešiť bezpečnostné incidenty,
čo môže zahŕňať spracovanie veľkých množstiev (možno citlivých) dát.} (z angl. \textit{Computer security incident response
team}).

Spomenuté agendy, stratégie a direktívy sa premietly aj v legislatíve Slovenska vo forme niekoľkých zákonov a vykonávacích
predpisov. V ďalšej kapitole rozoberieme relevantnú legislatívu, pričom by už mal byť čítateľovi jasnejší ich pôvod a zámer.
