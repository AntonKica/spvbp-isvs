\chapter{Všeobecný kontext}
Neoddeliteľnou súčasťou nášho každodenného života je informácia. Informáciu v súčasnej dobe uchovávame, spracovávame a 
šírime najmä v digitálnej forme. Práve digitálne informačné a komunikačné technológie (ďalej d-IKT) nám pomáhajú 
zaobchádzať s enormným množstvom informácií vo svete. Pre jemnú ilustráciu, jedna štatistika \cite{STATISTA} odhadla, 
že v roku 2020 sa vo svete nachádzalo vyše 79 ZB dát (ZB -- zetabajt, $1$ ZB = $10^9$ TB).


Aj samotná Európska únia (ďalej EÚ) sa snaží šíriť d-IKT a posilniť digitalizáciu členských štátov. EÚ vypracovala
digitálny rámec pod názvom \say{Europe’s Digital Decade}, vrámci ktorého boli vytýčené spoločné ciele digitálnej 
transformácie do roku 2030.

Od roku 2014 monitoruje Európska komisia digitálny rozvoj členských štátov pomocou Indexu digitálnej ekonomiky a
spoločnosti
\cite{DESI22}. V skratke, tento index profiluje štáty v štyroch oblastiach: pripojiteľnosť; ľudský kapitál; 
využívanie internetových služieb; integrácia digitálnych technológií.

Určite by pre nikoho nebolo prekvapením, že najväčší rozvoj v digitalizácií prebehol v roku 2020 - keď nastúpila pandémia COVID-19.
Priamo z výročnej správy DESI\cite{DESI22} o roku 2020:
\begin{quote}
  Businesses provided more fully digitised products and
  services: 34\% before the Covid-19 crisis and 50\% during the pandemic6; and bought more cloud
  computing services: 24\% before the pandemic in 2019 and 41\% in 2021. Significant differences
  continue to persist between large enterprises and SMEs\textsuperscript{8}, given that 72\% of large enterprises
  subscribed to cloud computing services compared to 40\% of SMEs.
\end{quote}

Pokrok v tomto smere sa v nasledujúcich rokoch znížil. Z indexu DESI\cite{DESI22} za rok 2022  má Slovensko skóre 43,4
(oproti európskemu priemeru 52,3), v poradí 24. (prvé Fínsko 69.6; posledné Rumunsko 30.6) spomedzi 27 štátov. Zaostávame.

\section{Kybernetická a informačná bezpečnosť}
Najprv si uvedieme základne pojmy podľa Malého výkladového slovníka \cite{KUDIaKB-MVS}: \\

\textbf{kybernetický priestor}
\begin{quote}
  globálny dynamický otvorený systém, ktorý tvoria
  telekomunikačné a počítačové siete,
  informačné a komunikačné systémy, ich
  programové vybavenie a údaje, ktoré sa
  pomocou nich spracovávajú; ...
\end{quote}


\textbf{kybernetická a informačná bezpečnosť} (ďalej KIB)
\begin{quote}
  systém opatrení na zaistenie odolnosti
  kybernetického priestoru, ako aj činností a
  prostriedkov zameraných na dosiahnutie
  požadovanej úrovne bezpečnosti prvkov
  kybernetického priestoru vrátane riešenia
  incidentov a následných opatrení a činností
\end{quote}

\textbf{aktívum}
\begin{quote}
  čokoľvek, čo má pre organizáciu hodnotu, čokoľvek, čo je: 
    hmotné (zariadenia, personál, ...) alebo nehmotné (peniaze, informácia, ...),
    môže sa stať objektom útoku a
    vyžaduje si ochranu
\end{quote}

\section{d-IKT na Slovensku}
Rôzne štátne inštitúcie, firmy, banky, poisťovne... využívajú d-IKT na svoju prevádzku. Pomocou d-IKT spravujú svoje 
systémy. V týchto systémoch sa nachádzajú mnohé aktíva. Tieto aktíva sú vystavené kybernetickému priestoru. 
V kybernetickom priestore sú prítomné rôzne hrozby -- prírodné, technické, ľudské.
Potrebujeme d-IKT, ich systémy a ich aktíva chrániť pred poruchami, výpadkami, chybami, odcudzením alebo útokmi v 
kybernetickom priestore.

\section{Potreba KIB}
Chrániť informácie nachádzajúce sa v systémoch verejnej správy nám, mimo morálnej potreby, ukladá jednak legislatíva 
Slovenska, no tiež legislatíva EÚ - mnoho z nás sa určite stretlo s nariadením GDPR (General Data Protection Regulation). 
Na Slovensku najprominentnejšie zákony a vyhlášky týkajúce sa KIB sú najmä:

\begin{itemize}
  \item Zákon o kybernetickej a informačnej bezpečnosti 69/2018
  \item Vyhláška NBÚ č. 362/2018
  \item Zákon o informačných technológiách verejnej správy (ITVS) 95/2019
  \item Vyhláška podpredsedu vlády 179/2020
\end{itemize}

Táto legislatíva kladie požiadavky KIB na informačné služby verejnej správy (ďalej ISVS). Požiadavky na 
vypracovanie bezpečnostného projektu, dokumentácie, riadenie KIB, analýzy a riadenia rizík, atď. pre ISVS. 

  \section{Súčasný stav KIB na Slovensku}
Národný bezpečnostný úrad (ďalej NBÚ) publikuje každoročne výročnú správu o KIB v Slovenskej republike. Spomedzi mnohé 
sa výročná správa venuje prehľadom globálnych hrozieb, incidentov, trendom a významným incidentom. Vykonáva bezpečnostné
audity v oblastiach: bankovníctvo, zdravotníctvo, energetika, infraštruktúra a verejná správa.

Podľa najnovšej výročnej správy \cite{KIBSK22} došlo za rok 2022 k $8'887'103$ potenciálnym bezpečnostným incidentom - naprieč všetkými sektormi. Keď sa zúžime na sektore verejnej správy (čo je predmetom našej bakalásrkej práce), audit v zistil
najmä tieto nedostatky: 

\begin{quote}
  \begin{itemize}
    \item nebol preukázaný systém riadenia kybernetickej bezpečnosti,
    \item bezpečnostná stratégia kybernetickej bezpečnosti ani ďalšia bezpečnostná dokumentácia nebola predložená,
    \item manažér kybernetickej bezpečnosti nie je formálne menovaný, je v konflikte záujmov, má nevhodne kumulované zodpovednosti,
    \item analýza rizík nie je zakotvená ako proces v interných predpisoch ani metodicky popísaná, nevykonáva sa,
    \item v organizácii sa nachádzajú vysoko privilegované účty, ktoré sú spoločné a nemajú definovaných vlastníkov a účel,
    \item v organizácii neexistuje definícia závažného kybernetického bezpečnostného incidentu,
    \item nevypracovala postupy a nemá dostatočné schopnosti na detekciu, zvládanie a poučenie sa z prípadných incidentov.
  \end{itemize}
\end{quote}

\section{Legislatíva}
Kybernetická a informačná bezpečnosť je obsiahnutá v niekoľkých zákonoch, resp. vyhláškach. V niektorých je KIB spomenutá okrajovo.
My sa sústredíme na tie zákony a vyhlášky, ktoré kladú podmienky a požiadavky na KIB v rámci organizácie. Konkrétne sú to 
69/2018\cite{69/2018}, 362/2018\cite{362/2018}, 95/2019\cite{95/2019} a 179/2020\cite{179/2020}. Pokúsime sa z nich vytýčiť
tie najrelevantnejšie informácie, keďže sú vcelku rozsiahle.

\subsection{Zákon č. 69/2018 Z. z., Zákon o kybernetickej bezpečnosti a o zmene a doplnení niektorých zákonov \cite{69/2018}}

V rámci tohto zákona sú vymedzené rôzne právomoci, kompetencie a povinnosti Národného bezpečnostného úradu (ďalej NBÚ). NBÚ spravuje
štandardy, určuje národnú stratégiu KIB, spravuje jednotky pre riešenie KIB incidentov, vykonáva audity, odníma certifikáty KIB, atď.

Tento zákon tiež vymedzuje minimálne požiadavky na zabezpečenie KIB, pričom nezasahuje do žiadnych detailov. 
Tieto požiadavky sú kladené na služby (a ich prevádzkovateľov), ktoré sú rozdelene na dve kategórie: základná služba a digitálna služba.

\textbf{Základnou službou} sa rozumie napr. prvok kritickej infraštruktúry alebo sektor bankovníctva, dopravy, energetika, verejná správa a
mnohé ďalšie (NBÚ udržiava kompletný zoznam), vrátane ich podsektorov. Tiež, služba závislá od základnej služby je považovaná za základnú službu.
Poskytovateľom základnej služby je orgán verejnej moci alebo fyzická osoba.

\textbf{Digitálnou službou} sa rozumie online trhovisko, internetový vyhľadávač alebo služba v oblasti cloud computingu. 
Poskytovateľom digitálnej služby je veľká organizácia aspoň o 50 zamestnancoch a ročným obratom vyše 10 mil. eur.

Nakoniec definuje 16 oblasti, pre ktoré sa majú realizovať bezpečnostné opatrenia a čo sa má byť v nich zahrnuté. 
Avšak, tento zákon nezachádza do žiadnych detailov bezpečnostných opatrení.   

\subsection{Vyhláška č. 362/2018 Z. z., Vyhláška Národného bezpečnostného úradu, ktorou sa ustanovuje obsah bezpečnostných opatrení, obsah a štruktúra
bezpečnostnej dokumentácie a rozsah všeobecných bezpečnostných opatrení \cite{362/2018}}

V tejto vyhláške sa detailnejšie špecifikuje obsah bezpečnostných opatrení a štruktúra bezpečnostnej dokumentácie.
Definuje stratégiu KIB, klasifikáciu a kategorizáciu informačných systémov. Detailnejšie popisuje bezpečnostných opatrení
pre všetky oblasti definovaných v \cite{69/2018}. 

Explicitne je však spomenuté, že 

\subsection{Zákon č. 95/2019 Z. z., Zákon o informačných technológiách vo verejnej správe a o zmene a doplnení niektorých zákonov \cite{95/2019}}
Tento zákon ustanovuje organizáciu správy ITVS, práva a povinnosti v oblasti ITVS a základné požiadavky kladené na ITVS a ich správu. Dva podstatné pojmy:

\begin{itemize}
  \item \textbf{informačná technológia} je prostriedok alebo postup slúžiaci na sprocovávanie údajov alebo informácií v elektronickej podobe.
  \item \textbf{informačná služba} je funkčný celok zabezpečujúci informačnú činnosť prostredníctvom technických a technologických prostriedkov.
\end{itemize}

Správu ITVS vykonáva orgán vedenia --- MIRRI a orgán riadenia --- ministerstvo, Generálna prokuratúra, obec, právnická osoba, atď.
