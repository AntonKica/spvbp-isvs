\documentclass{beamer}
\usepackage[utf8]{inputenc}
\usepackage[slovak]{babel}
\setbeamertemplate{navigation symbols}{}
\setbeamertemplate{footline}[frame number]

% only for this example, otherwise in .bib file
%\usepackage[style=verbose,backend=biber]{biblatex}
%\addbibresource{bibl.bib}

\title{Systém na podporu vypracovania bezpečnostných projektov pre malé ISVS}

\author{
    {Anton Kica} \\
    {\and} \\
    {\textit{Školiteľ}} \\
    {RNDr. Daniel Olejár, PhD.}
}

\institute{FMFI Matfyz}
\date{2024}

\begin{document}

\frame{\titlepage}

\begin{frame}
    \frametitle{Naša téma a cieľ}
    \begin{itemize}
      \item venujeme sa problematike informačnej bezpečnosti vo verejnej správe,
      \item našim cieľom je vytvoriť systém na riadenie informačnej bezpečností pre malé ISVS,
      \item tento systém má pomôcť zorientovať sa v legislatívnych požiadavkách a
      \item pomôcť vybudovať a prevádzkovať ISMS v organizácií
    \end{itemize}
\end{frame}

\begin{frame}
    \frametitle{Zatiaľ sme}
    \begin{itemize}
      \item oboznámili sme sa so súvisiacu legislatívou, zatiaľ nie do hĺbky a
      \item preštudovali sme súvisiacu literatúru
    \end{itemize}
\end{frame}
\begin{frame}
    \frametitle{Možné komplikácie}
    \begin{itemize}
      \item pre mňa osobne sú to neznáme vody,
      \item spracovanie požiadaviek a ich prezentácia používateľovi môže byť dosť komplexná a
      \item tvorba užívateľského rozhrania
    \end{itemize}
\end{frame}
\begin{frame}
    \frametitle{Čas, plán a blízka budúcnosť}
    ak počítame 2-3 týždne na etapu:
    \begin{itemize}
      \item detailná analýza legislatívy,
      \item porovnanie legislatívy so štandardami (z ktorých bola vytvorená),
      \item oboznámiť sa s procesom zavádzania ISMS a súvisiacich komplikácií,
      \item pomocou zozbieraných poznatkov a metodík vypracovať pomocný systému a nakoniec
      \item zdokumentovať funkcionalitu systému a vypracovať príručku systému
    \end{itemize}
\end{frame}
\end{document}
